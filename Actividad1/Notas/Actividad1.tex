\documentclass{article}

% set font encoding for PDFLaTeX or XeLaTeX
\usepackage{ifxetex}
\ifxetex
  \usepackage{fontspec}
\else
  \usepackage[T1]{fontenc}
  \usepackage[utf8]{inputenc}
  \usepackage{lmodern}
\fi

% used in maketitle
\title{Actividad 1}
\author{Cabello Lopez Marco Antonio\\
Departamento de Fisica \\
Universidad de Sonora}
\date{1 de Septiembre de 2017}


% Enable SageTeX to run SageMath code right inside this LaTeX file.
% documentation: http://mirrors.ctan.org/macros/latex/contrib/sagetex/sagetexpackage.pdf
% \usepackage{sagetex}

\begin{document}
\maketitle
\clearpage

\section{Introducción}
Esta actividad nos describe como leer un manual de Linux en el nos enseñan\\
una forma de utilizar la terminal y sus conceptos básicos, asi como  \\
una introducción a los comandos Bash que son la base de un optimo uso \\
de Linux.
\subsection{Comandos Bash}
Son un conjunto de parametros utilizados para la administración y\\
configuración del sistema, asi como un conjunto de combinaciones \\
especiales de teclas para realizar tareas en entornos Linux      \\
mediante un interperete de comandos Bash.


\subsubsection{Ejemplos de comandos de Bash}
\begin{verbatim}
Comando		Descripcion
ls		Enlista los  archivos dentro de un directorio.
		Ej. ls -l /home/cabello1/ProgFortran

echo		Es usado para para mandar mensajes a la terminal o archivos.
		Ej. echo hola mundo

pwd		Es usado para saber el directorio en el que te encuentras
		trabajando.

file		Te dice de que tipo es un archivo en particular
		Ej. file notas.txt

man		Abre el manual del comando que pones despues de man
		Ej. man ls

mkdir		Este comando crea un directorio en la ubicacacion que te
		encuentras.
		Ej. mkdir ProgFortran

rmdir		Remueve/elimina el directorio que nombras.
		Ej. rmdir ProgFortran

rm		Remueve/elimina el archivo que nombres.
		Ej. rm notas.txt

touch		Si se pone un nombre de archivo que no exista, creara un
		archivo en blanco con ese nombre.

cp		Copia un archivo o directorio en el lugar indicado.
		Ej. cp notas.txt /home/cabello1/ProgFortran

mv		Mueve un archivo o directorio al lugar indicado.
		Ej. mv notas.txt/home/cabello1/ProgFortran/Notas

vi		Editas el archivo que menciones.
		Ej. vi notas.txt
		Puedes cambiar de la funcion "editar" a funcion
		"insertar" con la tecla "i".

cat		Este comando hace visible el texto del archivo.
		Ej. cat notas.txt

less		Hace visible el contenido de un texto, se
		recomienda para textos mas grandes ya que
		es posible moverte entre hojas.
		Ej. less notas.txt

chmod		Cambia los permisos que tiene una persona respecto
		a un archivo.
		Primero va el nombre del usuario
		+ o - para indicar si remueves o das permiso.
		r para que tengan permisos de leer
		w si tendran permisos de escribir
		x si se les otorgara permiso de ejecutar
		Ej. chmod cabello1+w

		
		

\end{verbatim}
\section{Bibliografía}
http://ryanstutorials.net/linuxtutorial/


\end{document}
