\documentclass{article}

% set font encoding for PDFLaTeX or XeLaTeX
\usepackage{ifxetex}
\ifxetex
  \usepackage{fontspec}
\else
  \usepackage[T1]{fontenc}
  \usepackage[utf8]{inputenc}
  \usepackage{lmodern}
\fi

% used in maketitle
\title{Actividad 2}
\author{Cabello Lopez Marco Antonio \\
 Departamento de Fisica \\
 Universidad de Sonora}
\date{13 de Septiembre del 2017}

% Enable SageTeX to run SageMath code right inside this LaTeX file.
% documentation: http://mirrors.ctan.org/macros/latex/contrib/sagetex/sagetexpackage.pdf
% \usepackage{sagetex}

\begin{document}
\maketitle
%comentarios

\section{Introduccion}
Movimiento de Proyectiles
\subsection{Archivo proyectil}
Utilizamos esté programa para obtener datos generales sobre el movimiento de proyectiles.
\begin{verbatim}
program projectile
  implicit none

  ! definimos constantes
  real, parameter :: g = 9.8
  real, parameter :: pi = 3.1415927

  ! definimos las variables
  real :: a, t, u, x, y
  real :: theta, v, vx, vy

  ! Leer valores para el ángulo a, el tiempo t, y la velocidad inicial u desde la terminal
  write(*,*) 'Dame el ángulo, el tiempo y la rapidez inicial'
  read(*,*) a, t, u

  ! convirtiendo ángulo a radianes
  a = a * pi / 180.0
  
  ! las ecuaciones de la posición en x y y
  x = u * cos(a) * t
  y = u * sin(a) * t - 0.5 * g * t * t

  ! La velocidad al tiempo t
  vx = u * cos(a)
  vy = u * sin(a) - g * t
  v = sqrt(vx * vx + vy * vy)
  theta = atan(vy / vx) * 180.0 / pi
 
 ! escribiendo el resultado en la pantalla
  write(*,*) 'x: ',x,'  y: ',y
  write(*,*) 'v: ',v,'  theta: ',theta

end program projectile
\end{verbatim}
\subsubsection{Ejemplo del programa}
Para $a$=$45$, $t$=$1$, y $u$=$31$ nos dio los datos:
$x$ = $21.9203091$, $y$ = $17.0203094 $, $v$ = $25.0479908$, y $\Theta$ = $28.9393463$
\subsection{Programa del Tiempo de vuelo}
Este programa lo utilizamos para obtener el tiempo de vuelo.
\begin{verbatim}
program tiempovuelo
  implicit none

  ! definimos constantes
  real, parameter :: g = 9.8
  real, parameter :: pi = 3.1415927

  ! definimos las variables
  real :: a, v0, tv
 

  ! Leer valores para el ángulo a, y la velocidad inicial v desde la terminal
  write(*,*) 'Dame el ángulo y la rapidez inicial'
  read(*,*) a, v0

  ! convertimos de grados a radianes
  a = a * pi / 180.0

  ! Ecuacion del tiempo de vuelo
  tv = 2.0 * v0 * sin(a) / g

  ! escribimos el resultado en la pantalla
  write(*,*) 'tiempo de vuelo es = ', tv

end program tiempovuelo
\end{verbatim}
\subsubsection{Ejemplo del programa}
Para $a$=$45$ y $v0$=$10$ nos dio un tiempo de vuelo $tv$=$1.44307506 $
\subsection{Programa de la Altura maxima}
Utilizamos este programa para conseguir la altura maxima.
\begin{verbatim}
program alturamaxima
  implicit none

  ! definimos constantes
  real, parameter :: g = 9.8
  real, parameter :: pi = 3.1415927

  ! definimos variables
  real :: a, v0, h

  write(*,*) 'Dame el ángulo, y la velocidad inicial'
  read(*,*) a, v0

  ! convirtiendo angulo a radianes
  a = a * pi / 180.0

  ! ecuacion de la velocidad final para la altura maxima
  h= (v0**2 * (sin (a))**2/(2 * g))
  
  ! escribiendo el resultado en la pantalla
  write(*,*) 'la altura maxima es = ', h, ' metros'

end program alturamaxima
\end{verbatim}
\subsubsection{Ejemplo del programa}
Para $a$=$30$, y $v0$=$21$ nos dio una altura maxima $h$=$5.62500000$
\subsection{Programa de la Trayectoria maxima}
Utilizamos este programa para obtener la trayectoria maxima en $x$
\begin{verbatim}
program xmax
  implicit none

  ! definimos constantes
  real, parameter :: g = 9.8
  real, parameter :: pi = 3.1415927

  ! definimos las variables
  real :: a, v0, x
  
  write(*,*) 'Dame el ángulo y la velocidad inicial'
  read(*,*) a, v0

  ! convirtiendo angulo a radianes
  a = a * pi / 180.0

  ! ecuacion de la trayectoria maxima
  x = (v0**2 * (sin (2 * a)))/g

  ! escribiendo el resultado en la pantalla
  write(*,*) 'la trayectoria maxima en x es = ', x, 'metros'

end program xmax
\end{verbatim}
\subsubsection{Ejemplo del programa}
Para $a$=$15$, y $v0$=$60$ nos da una trayectoria maxima $x$=$183.673462$
\subsection{Conclusión}
Estos programas son muy utiles al momento de calcular valores en el movimiento del tiro parabolico y de ejemplo para la programacion de  los mismos.
\subsection{Bibliografia}
$https://en.wikipedia.org/wiki/Projectile_motion$
\end{document}
