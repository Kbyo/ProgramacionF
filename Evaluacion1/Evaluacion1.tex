\documentclass{article}

% set font encoding for PDFLaTeX or XeLaTeX
\usepackage{ifxetex}
\ifxetex
  \usepackage{fontspec}
\else
  \usepackage[T1]{fontenc}
  \usepackage[utf8]{inputenc}
  \usepackage{lmodern}
\fi

% used in maketitle
\title{Evaluacion 1}
\author{Cabello Lopez Marco Antonio\\
Departamento de Fisica \\
Universidad de Sonora}
\date{Hermosilo, Sonora a Lunes 30 de octubre del 2017}


% Enable SageTeX to run SageMath code right inside this LaTeX file.
% documentation: http://mirrors.ctan.org/macros/latex/contrib/sagetex/sagetexpackage.pdf
% \usepackage{sagetex}

\begin{document}
\maketitle
\clearpage

\section{Problema 1: Esfera}
\begin{verbatim}
program ESFERA

! Declaracion de tipo de las variables.
  IMPLICIT NONE   

  integer :: ierr
  character(1) :: yn
  real :: RADIO, AREA, VOLUMEN
  real, parameter :: pi = 3.141592653589793

  interactive_loop: DO

! Impresion en pantalla.
    WRITE (*,*) 'Enter radius of the sphere..'
    READ (*,*,IOSTAT=ierr) RADIO

! Ingreso de datos.
    IF (ierr /= 0) THEN
      WRITE(*,*) 'Error, invalid input.'
      CYCLE interactive_loop
    END IF

! Calculos.
    AREA = 4*pi * (RADIO**2)
    VOLUMEN= (4/3)*pi*(RADIO**3)

! Impresion del resultado.
    WRITE (*,'(1x,a7,f6.2,5x,a7,f6.2,5x,a5,f6.2)') &
      'RADIO=',RADIO,'AREA=',AREA,'VOLUMEN=',VOLUMEN

    yn = ' '
    yn_loop: DO
      WRITE(*,*) 'Perform another calculation? y[n]'
      READ(*,'(a1)') yn
      IF (yn=='y' .OR. yn=='Y') EXIT yn_loop
      IF (yn=='n' .OR. yn=='N' .OR. yn==' ') EXIT interactive_loop
    END DO yn_loop

  END DO interactive_loop

! Finalizacion del programa.
END PROGRAM ESFERA

\end{verbatim}

\section{Problema 2: Medias}
\begin{verbatim}
PROGRAM MEDIAS

!Declaracion de las variables
IMPLICIT NONE
INTEGER :: sumatory, x, tell
REAL :: SUMATORIAARITMETICA, SUMATORIAARMONICA, harmony, fx, fa, fb

!Condiciones del programa
PRINT*, "Este programa realiza las medias de una sumatoria, presione 0 para finalizar el proceso"
OPEN(UNIT=10, FILE="SumData.DAT", STATUS='unknown')

sumatory = 0
tell = 0
harmony = 0

!Desarrollo del programa
 DO
  PRINT*, "Add:"
  READ*, x
  IF (x == 0) THEN
   EXIT
 ELSE
sumatory = sumatory + x
tell = tell + 1
fx = float(x)
fx = 1/fx
harmony = harmony + fx

 END IF
 WRITE(10,*) x
 END DO
fb = float(sumatory)
fa = float(tell)
SUMATORIAARITMETICA = fb / fa
SUMATORIAARMONICA = fa / harmony


PRINT*, "Sumatoria =", sumatory
WRITE(10,*) "Sumatoria =", sumatory
WRITE(10,*)' '
PRINT*, "Media aritmetica =", SUMATORIAARITMETICA
WRITE(10,*) "Media aritmetica =", SUMATORIAARITMETICA
WRITE(10,*) ' '
PRINT*, "Media armonica =", SUMATORIAARMONICA
WRITE(10,*) "Media armonica =", SUMATORIAARMONICA
WRITE(10,*) ' '


close(10)

! Finalizacion del programa.
END PROGRAM MEDIAS


\end{verbatim}

\section{Problema 3: Leibniz}
\begin{verbatim}
PROGRAM LEIBNIZ

!Declaracion de las variables
 IMPLICIT NONE
INTEGER :: i
REAL :: n, iteracion, pi

!Condiciones del programa
  pi = 1
  iteracion = 1
 WRITE(*,*) 'El valor de pi/4 segun las repeticiones:'

!Desarrollo del programa
     DO i=1, 50
     iteracion = iteracion * (-1)
     n = 2 * i + 1
     n = 1 / n
     n = n * (iteracion)
     pi = pi + n
	IF (i.EQ.10) THEN
	WRITE(*,*) ' '
	WRITE(*,*) '10:', pi
        END IF

	IF (i.EQ.20) THEN
	WRITE(*,*) ' '
	WRITE(*,*) '20:', pi
        END IF

	IF (i.EQ.30) THEN
	WRITE(*,*) ' '
	WRITE(*,*) '30:', pi
        END IF

	IF (i.EQ.40) THEN
	WRITE(*,*) ' '
	WRITE(*,*) '40:', pi
        END IF

	IF (i.EQ.50) THEN
	WRITE(*,*) ' '
	WRITE(*,*) '50:', pi
        END IF

END DO

! Finalizacion del programa.
END PROGRAM LEIBNIZ

\end{verbatim}


\end{document}
